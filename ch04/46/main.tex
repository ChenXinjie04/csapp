\documentclass{article}
\usepackage{enumitem}

\begin{document}
\begin{enumerate}[label=\textbf{\Alph*.}]
	\item In light of analysis done in Practice Proble 4.8, does this code sequence
	correctly describe the behavior of the instruction \texttt{popq \%rsp}? Explain.
	No, \texttt{popq \%rsp} will set the \%rsp register with the value at the top of the stack.
	\item How could you rewrite the code sequence so that it correctly describes both the
	cases where REG is \%rsp as well as other register? \\
	\texttt{
		addq \$8, \%rsp \\
		movq 8(\%rsp), REG \\
	}
\end{enumerate}
\end{document}
