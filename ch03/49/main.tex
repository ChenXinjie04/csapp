\documentclass{article}
\usepackage{enumitem}

\begin{document}
\begin{enumerate}[label=\textbf{\Alph*.}]
	\item \textbf{Explain, in mathematical terms, the logic in the computation of $s_2$ on lines
	$5-7$. \textit{Hint:} Think about the bit-level representation of -16 and its effect
	in the \textit{andq} instruction of line 6.} \\
	The \textit{leaq} instruction of line $5$ computes the value $8n+22$, which is
	then rounded down to the nearest multiple of $16$ by the \textit{andq} instruction of line $6$.
	The resulting value will be $8n+8$ when $n$ is odd and $8n+16$ when $n$ is even,
	and this value is subtracted from $s_1$ to give $s_2$.
	\item \textbf{Explain, in mathematical terms, the logic in the computation of $p$ on lines 
	8-10.} \\
	The three instructions in this sequence round $s_2$ up to the nearest
	multiple of $8$. They make use of the combination of biasing and shifting that we
	saw for dividing by a power of 2 in Section 2.3.7. \\
	\item
	\item 
	We can see that $s_2$ is computed in a way that preserves whatever offset $s_1$ has
	with the nearest multiple of $16$. We can also see that $p$ will be aligned on
	a multiple of $8$, as is recommended for an array of 8-byte elements.
\end{enumerate}
\end{document}
