\documentclass{article}
\usepackage{enumitem}

\begin{document}
\begin{enumerate}[label=\textbf{\Alph*.}]
	\item
\begin{tabular}{|c c c c c c c c| c}
	\hline
	00 & 00 & 00 & 00 & 00 & 40 & 00 & 76 & Return address \\
	\hline
	01 & 23 & 45 & 67 & 89 & AB & CD & EF & Saved \%rbx \\
	\hline
	&&&&&&&&\\
	\hline
	&&&&&&&&\\
	\hline
\end{tabular}
	\item \textbf{Modify your diagram to show te effect of the call to gets(line 5).}
\begin{tabular}{|c c c c c c c c| c}
	\hline
	00 & 00 & 00 & 00 & 00 & 40 & 00 & 34 & Return address \\
	\hline
	33 & 32 & 31 & 30 & 39 & 38 & 37 & 36 & Saved \%rbx \\
	\hline
	35 & 34 & 33 & 32 & 31 & 30 & 39 & 38 & $\leftarrow$ buf = \%rsp \\
	\hline
	37 & 36 & 35 & 34 & 33 & 32 & 31 & 30 & $\leftarrow$ buf = \%rsp \\
	\hline
\end{tabular}
	\item \textbf{To what address does the program attempt to return?} \\
	The program is attempting to return to address 0x400034. The lower-order two bytes
	were overwritten by the code for character '4' and terminating null character.
	\item  \textbf{What register(s) have corrupted value(s) when get\_line returns?} \\
	The saved value for register \%rbx will be loaded into the register before
	\textit{get\_line} returns.
	\item \textbf{Besides the potential for buffer overflow, what other things are wrong
	with the code for \textit{get\_line}?} \\
	The call to malloc should have had \textit{strlen(buf)+1} as its argument, and
	the code should also check that the returned value is not equal to \textit{NULL}.
\end{enumerate}
\end{document}
