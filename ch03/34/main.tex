\documentclass{article}
\usepackage{enumitem}
\setlength{\parindent}{0pt}

\begin{document}
\begin{enumerate}[label=\textbf{\Alph*.}]
	\item \textbf{Identify which local values get stored in callee-saved registers.} \\
	From line 9 to line 14, we can see that the local values a0-a5 are stored into
	callee-saved registers \%rbx, \%r15, \%r14, \%r13, \%r12 and \%rbp, respectively.
	\item \textbf{Identify which local values get stored on the stack.} \\
	Local values a6 and a7 are stored on the stack at offset 0 and 8 relative to the
	stack pointer.
	\item \textbf{Explain why the program could not store all of the local values in callee-saved
	registers.} \\
	After storing six local variables, the program has used up the supply of callee-saved
	registers. It stores the two local values on the stack.
\end{enumerate}
\end{document}
