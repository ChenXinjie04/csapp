\documentclass{article}
\usepackage{enumitem}

\begin{document}
\begin{enumerate}[label=\textbf{\Alph*.}]
	\item \textbf{What are the byte offsets of all the fields in the structure?}
	\begin{tabular}{c c c c c c c c c}
		Field & a & b & c & d & e & f & g & h \\
		\hline
		size & 8 & 2 & 8 & 1 & 4 & 1 & 8 & 4 \\
		offset & 0 & 8 & 16 & 24 & 28 & 32 & 40 & 48 \\
	\end{tabular}
	\item \textbf{What is the total size of the structure?}
	The total size of the structure is 56 bytes. The end of the structure must be padded
	with 4 bytes to satisfy the 8 bytes alignment.
	\item \textbf{Rearrange the fields of the structure to minimize wasted space,
	and then show the byte offsets and total size of the rearranged structure.} \\
	One strategy that works, when all data element have a length equal to a power of 2,	
	is to order the structure elements in descending order of size. This leads to a
	declaration \\
	struct \{ \\
		char *a; \\
		double c; \\
		long g; \\
		float e; \\
		int h; \\
		short b; \\
		char d; \\
		char f; \\
	\} \\
	with the following offsets:
	\begin{tabular}{c c c c c c c c c}
		Field & a & c & g & e & h & b & d & f \\
		\hline
		size & 8 & 8 & 8 & 4 & 4 & 2 & 1 & 1 \\
		offset & 0 & 8 & 16 & 24 & 28 & 32 & 34 & 35 \\
	\end{tabular}
	The structure must be padded by 4 bytes to satisfy the 8-byte alignment requirement,
	giving a total of 40 bytes.
\end{enumerate}
\end{document}
